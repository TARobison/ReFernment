\PassOptionsToPackage{unicode=true}{hyperref} % options for packages loaded elsewhere
\PassOptionsToPackage{hyphens}{url}
%
\documentclass[]{article}
\usepackage{lmodern}
\usepackage{amssymb,amsmath}
\usepackage{ifxetex,ifluatex}
\usepackage{fixltx2e} % provides \textsubscript
\ifnum 0\ifxetex 1\fi\ifluatex 1\fi=0 % if pdftex
  \usepackage[T1]{fontenc}
  \usepackage[utf8]{inputenc}
  \usepackage{textcomp} % provides euro and other symbols
\else % if luatex or xelatex
  \usepackage{unicode-math}
  \defaultfontfeatures{Ligatures=TeX,Scale=MatchLowercase}
\fi
% use upquote if available, for straight quotes in verbatim environments
\IfFileExists{upquote.sty}{\usepackage{upquote}}{}
% use microtype if available
\IfFileExists{microtype.sty}{%
\usepackage[]{microtype}
\UseMicrotypeSet[protrusion]{basicmath} % disable protrusion for tt fonts
}{}
\IfFileExists{parskip.sty}{%
\usepackage{parskip}
}{% else
\setlength{\parindent}{0pt}
\setlength{\parskip}{6pt plus 2pt minus 1pt}
}
\usepackage{hyperref}
\hypersetup{
            pdftitle={Why ReFernment},
            pdfborder={0 0 0},
            breaklinks=true}
\urlstyle{same}  % don't use monospace font for urls
\setlength{\emergencystretch}{3em}  % prevent overfull lines
\providecommand{\tightlist}{%
  \setlength{\itemsep}{0pt}\setlength{\parskip}{0pt}}
\setcounter{secnumdepth}{0}
% Redefines (sub)paragraphs to behave more like sections
\ifx\paragraph\undefined\else
\let\oldparagraph\paragraph
\renewcommand{\paragraph}[1]{\oldparagraph{#1}\mbox{}}
\fi
\ifx\subparagraph\undefined\else
\let\oldsubparagraph\subparagraph
\renewcommand{\subparagraph}[1]{\oldsubparagraph{#1}\mbox{}}
\fi

% set default figure placement to htbp
\makeatletter
\def\fps@figure{htbp}
\makeatother


\title{Why ReFernment}
\date{}

\begin{document}
\maketitle

\hypertarget{why-refernment}{%
\subsection{Why ReFernment}\label{why-refernment}}

ReFernment was created in response to the drudgery of manually
annotating a plastome sequence that has levels of RNA editing. These
editing sites result in a genomic sequence that contains features that
look like errors (e.g., internal stop codons) so the annotations will be
rejected by NCBI unless they are anotated correctly. ReFernment corrects
the annotation to produce files ready for GenBank submission ReFernment
takes as input a GenBank flatfile (.gb) and a gff file and generates new
annotations for nonsense mutations that can reasonably be explained by
RNA editing and provides conceptual translations for coding sequences
with RNAediting. It should be made clear that ReFernment is not intended
to predict every RNA editing site, as can other available software
packages. These packages rely on RNA seq data to compare the the genomic
DNA, and thus determine if a nucleotide has been edited, but these data
are not always available to researchers. We made ReFernment as a simple
tool to save time for those who don’t have RNA seq data available to
them. We hope that ReFernment also saves the time of GenBank staff, as
well has helping improve the quality of annotations available in
GenBank.

\hypertarget{how-it-works}{%
\subsection{How it works}\label{how-it-works}}

ReFernment operates by refining existing annotations. What this means is
that ReFernment uses an annotation generated by programs such as DOGMA,
CpGAVAS, Verdant or AGORA (Wyman et al. 2004; Liu et al. 2012; McKain et
al. 2017; Jung et al. 2018), and adjusts these annotations to account
for RNA editing. The basic operation or ReFernment is extremely simple.
First, ReFernment checks the start and stop codons of each gene. In both
cases ReFernment initially checks whether the codon is a valid start or
stop, if the codon is not valid it checks whether an RNA editing event
would result in the restoration of the codon to a valid start or stop
(e.g. ACG -\textgreater{} AUG). If the codon is not valid, even after
checking for possible RNA editing, ReFernment checks whether nearby
codons (within 5 nucleotides) represent valid codons; if so, refernment
changes the gene boundaries to start or stop at those valid sites. Next,
ReFernment checks whether a gene has any internal stops, and if so,
checks whether RNA editing would restore these nonsense translations,
adjusting the translation on the GenBank flatfile to account for this.
Finally, ReFernment edits the imputed GenBank flat file (appendix),
adding annotations indicating the sites where RNA editing occurred with
‘misc\_feature’ flags, adding necessary RNA editing flags to the
relevant genes, and providing a conceptual translation for each gene.

ReFernment operates under the assumption that only U-to-C or C-to-U RNA
editing is occuring in the plastome. Additionally, ReFernment assumes
that all nonsense mutations are the result of RNA editing. Since most of
the genes that reside within the plastome are vital to photosynthetic
function, it is assumed that these genes will remain operational. There
may be cases where internal stops, bad starts, or missing stops are
actually the result of a real mutation, especially in parasitic lineages
(citation maybe rafflasia). In most cases, however, ReFerment
annotations should reflect real RNA editing sites. A major limitation of
ReFerment is that the annotations it produces are only as good as the
annotations it is provided. If a gene annotation is frameshifted, if
there are assembly errors, or if an annotation has the incorrect start
and stop sites ReFernment will not know. In other words, ReFerment is
not a substitute for manually checking gene annotations, nor is
ReFerment a fix for sloppy annotation. If there are more than 5 detected
internal stops in a gene ReFernment will suggest to the user that they
manually check that gene. There are cases where real genes see more than
5 RNA edited internal stops, but these are relatively rare, so users
should use best judgement.

\hypertarget{how-to-use-it}{%
\subsection{How to use it}\label{how-to-use-it}}

ReFernment requires both a GenBank flatfile and a gff file (without
sequence), to produce annotations for each genome. ReFernment takes as
input the following:

\begin{itemize}
\tightlist
\item
  \texttt{gbFolderPath}: directory containing the gb files
\item
  \texttt{gffFolderPath}: directory containing the gff files
\item
  \texttt{outputFolderPath}: the directory you would like to output the
  newly generated GenBank flatfiles
\item
  \texttt{genomes}: the names of the files being used
\end{itemize}

The \texttt{gbFolderPath} and \texttt{gffFolderPath} can refer to the
the same directory, but unless you are \emph{totally confortable} with
ReFernment overwriting the original gb files, then
\texttt{outputFolderPath} should refer to a directory seperate from the
other two. When preparing to use ReFernment you should keep the names of
the gb files and gff files identical, except for their extensions. For
example \texttt{Hemionitis\_subcordata.gb} and
\texttt{Hemionitis\_subcordata.gff}. This makes it easy for refernment
to loop through many files all at once if you have large numbers of
plastomes that need to be annotated. Along those lines, \texttt{genomes}
should not contain the file extension. ReFernment will figure add the
file extensions later.

Below we want ReFernment to annotate the the plastomes of Asplenium
pekinense and Woodwardia unigemmata. To start off, we declare a vector
named \texttt{genomes} which contains all of the plastomes which you
would like to annotate.

\begin{verbatim}
genomes <- c("Asplenium_pek", "Woodwardia_uni")
\end{verbatim}

Next, we'll declare another vector containing the path to both the input
folders (\texttt{gb} and \texttt{gff})

\begin{verbatim}
gbFolder <- "C:\\Users\\Me\\Location\\Of\\GB\\Files\\"

gffFolder <- "C:\\Users\\Me\\Location\\Of\\gff\\Files\\"

outputFolder <- "C:\\Users\\Me\\Location\\Of\\Output\\Folder\\"
\end{verbatim}

Note that if we wanted to \texttt{gbFolder} and \texttt{gffFolder} could
refer to the same directory

\begin{verbatim}
gbFolder <- "C:\\Users\\Me\\Location\\Of\\input\\Files\\"

gffFolder <- "C:\\Users\\Me\\Location\\Of\\input\\Files\\"

outputFolder <- "C:\\Users\\Me\\Location\\Of\\Output\\Folder\\"
\end{verbatim}

Finally, we simply call the \texttt{ReFernment} function and wait for it
to finish. This can take several minutes if you have a large number of
plastomes.

\begin{verbatim}
ReFernment(gbFolder, gffFolder, outputFolder, genomes)
\end{verbatim}

As ReFernment runs, it may produce the following warning message:

\begin{verbatim}
There are a high number of edited Stops ( [number] ) in [geneName] manually check to make sure frame is correct
\end{verbatim}

This message is intended to warn the user of two common problems:

\begin{enumerate}
\tightlist
\item
  potential assembly errors that would result in an entire gene
  appearing to be frameshifted
\item
  annotation errors where the proper frame for a gene was not selected
  to begin with
\end{enumerate}

The warning will appear if a given coding sequence has more than 5
internal stops. There are, of course, cases where this happens and it is
not the result of assembly error, but these are relatively rare.
Especially if there are more than 10 internal stops the user should take
a close look at this coding sequecne to check for assembly error or
annotation errors.

\hypertarget{conclusions}{%
\subsection{Conclusions}\label{conclusions}}

While other programs such as PREPACT or \textbf{ChloroSeq} are excellent
at predicting RNA editing sites given the user has access to cDNA, they
do not provide RNA editing for users who don’t have such data.
ReFernment offers a viable alternative when cDNA is not available.
Furthermore, ReFernment offers easy annotation of RNA edited sites and
automatic conceptual translation, easing the process of GenBank
submission and saving the user valuable time.

\end{document}
